\chapter{Conclusion and Future Work}\label{chp4}

To summarize, we have proposed novel multilevel Bayesian joint models for personalized prediction of PE risk in hierarchically structured CF data. The joint model is in the sense of joint analysis of two types of outcomes simultaneously, which including ppFEV1 as a biomarker and PE as a binary event. Specifically, we account for PE in two submodel frameworks: i) GLMM; ii) survival model. In the context of GLMM, we demonstrate that the center with more severe CF patients tends to have higher risk of PE and patients with worse averaged lung function are more likely to experience PE. Whilst with respect to survival model, every one percentage predicted increase in 'true and unobserved' ppFEV1 would decrease 3.92\%, 4.88\%, 3.92\%, 3.92\%, 2.96\%, 3.92\% for the PE risk from selected Center 1 to Center 6. The novelty of this dissertation is evident, such as center-specific power parameter, center-specific parameter for the association structure and baseline hazard function, individual-specific prediction start time and customized applicable Stan programs.

In both chapters, we have implemented 80-20\% cross-validation. However, a recently published paper proposed sample size guidelines for developing a clinical prediction model \cite{Riley2020}. The article recommended resampling methods (such as bootstrapping) as the internal validation with at least 10 events for each predictor parameter. Another potential issue with our employed \emph{Stepwise} algorithm has been raised in a recent paper by \cite{Harhay2020}. They suggested that such procedure might lead to bias and overfitting problems. The backwards variable selection algorithm with a large significant level seems to be a better approach (\cite{Heinze2017}). From the clinical perspective in our application context, \emph{Stepwise} approach provides a straightforward means of feature selection, and from the statistical point of view, predictors have minimal impacts on predictive performance as long as the covariance structure is well specified. The past and current work highlight the need to proceed carefully about intended purpose of the prediction model while prioritizing feature selection elements as appropriate.  

Additional sources of bias might be brought by left-truncation within CF registry data. Although we remedy it by adding birth cohort as a covariate, this topic might be of interest to be devised by referring to \cite{Krol2016} and \cite{Piccorelli2012}. Given the fact that joint model comes at a cost of intensive computational time, as suggested in \cite{Barrett2019}, a possible modeling strategy can be to use the two-stage method for model selection and the joint model for inference. Moreover, an embarrassing parallel-MCMC algorithm mentioned in \cite{Ren2021} might be helpful to reduce running time. Finally, the RShiny app (\cite{shiny2013}) is under programming with the aim to facilitate the individual (dynamic) prediction from our proposed joint models.  

 % There are many widely applied R packages for GLMM, namely, lme4 by Bates et al. (2015), MCMCglmm by Hadfield 2010, brms by Burkner (xx), GLMMadaptive by Rizopoulos xx. Among which brms is for Bayesian GLMM using Stan. Though these packages more or less support variable existing link functions, our customized link function is not yet incorporated. Therefore, we code by ourselves via user friendly interfact rstan.  

